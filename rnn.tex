\documentclass{paper}
\usepackage[utf8]{inputenc}
\usepackage{bm}
\usepackage{natbib}
\usepackage{amsmath}
\usepackage{hyperref}
\title{Short note on recurrent neural networks}
\author{Grzegorz Chrupała}

\begin{document}
\maketitle
\section{Simple Recurrent Nets}
Simple recurrent nets (SRN) were introduced by
\cite{elman1990finding,elman1991distributed}:
  \begin{equation}
    \label{eq:srn}
    h_t = \sigma(Wx_t + Uh_{t-1})
  \end{equation}
where $\sigma$ is an elementwise nonlinearity such as the inverse
logit:
\[
\sigma(z) = \frac{1}{1+\exp(-z)}
\]

\section{Long Short-term Memory}
Long Short-Term Memory networks were introduced by
\cite{hochreiter1997long} in order to remedy problems  with the
backpropagation in SRN. The formulation is the following:

\begin{align*}
  i_t = & \sigma(W_{i}x_t + U_{i}h_{t-1}) \\
  f_t = & \sigma(W_{f}x_t + U_{f}h_{t-1}) \\
  o_t = & \sigma(W_{o}x_t + U_{o}h_{t-1}) \\
  c_t = & f_t \odot c_{t-1} + i_t \odot \tanh(W_{c}x_t + U_{c}h_{t-1}) \\
  h_t = & o_t \odot c_t \\
\end{align*}
where $\odot$ is elementwise multiplication.
As can be seen, at each time step, LSTM has two separate states: the
hidden state $h_t$, and the memory cell $c_t$. Three gates control whether
to forget the current cell value ($f$), whether it should read its input
($i$)  and whether to output the new cell value ($o$). This video
lecture by Hinton has a good explanation of LSTMs: \url{https://www.youtube.com/watch?v=lsV5rFbs-K0&list=PLnnr1O8OWc6YM16tj9pdhBZOS9tDktNrx&index=5}
\section{Gated Recurrent Units}
Gated Recurrent
Units (GRU) were first introduced by \cite{cho2014properties} and
\cite{chung2014empirical} as a simpler alternative to LSTMs.
In a GRU, activation at time $t$ is the linear combination of previous
activation, and candidate activation:
\begin{align*}
   h_t =& (1 - z_t)\odot h_{t-1} + z_t \odot \tilde{h}_t \\
   z_t =& \sigma_s(W_z x_t + U_z h_{t-1}) \\
   \tilde{h}_t =& \sigma(W x_t + U(r_t \odot h_{t-1})) \\
   r_t =& \sigma_s(W_r x_t + U_r h_{t-1})
\end{align*}
The partially linear
dependence between current and previous state makes it easier for
these networks to be trained via backpropagation.


\bibliography{biblio}
\bibliographystyle{apalike}
\end{document}
